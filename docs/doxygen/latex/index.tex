\hyperlink{namespaceEBGeometry}{E\+B\+Geometry} is a code for computing signed distance functions to watertight and orientable surface grids. It was written to be used with embedded-\/boundary (EB) codes like Chombo or A\+M\+ReX.

Tesselations must consist of planar polygons (not necessarily triangles). Internally, the surface mesh is stored in a doubly-\/connected edge list (D\+C\+EL), i.\+e. a half-\/edge data structure. On watertight and orientable grids, the distance to any feature (facet, edge, vertex) is well defined, and can naively be computed in various ways\+:


\begin{DoxyItemize}
\item Directly, by iterating through all facets.
\item With conventional bounding volume hierarchies (B\+V\+Hs).
\item With compact (linearized) B\+V\+Hs.
\end{DoxyItemize}

The B\+V\+Hs in \hyperlink{namespaceEBGeometry}{E\+B\+Geometry} are not limited to facets. Users can also embed entire objects (e.\+g., analytic functions) in the B\+V\+Hs, e.\+g. the \hyperlink{namespaceBVH}{B\+VH} accelerator can be used to accelerate the signed distance computation when geometries contain many objects. B\+V\+Hs can also be nested so that the \hyperlink{namespaceBVH}{B\+VH} accelerator is used to embed objects that are themselves contained in a \hyperlink{namespaceBVH}{B\+VH}. For example, a scene consisting of many objects described by surface grids can be embedded as a B\+V\+H-\/of-\/\+B\+VH type of scene. In addition, \hyperlink{namespaceEBGeometry}{E\+B\+Geometry} provides standard operators for signed distance fields like rotations, translations, and scalings.



\subsection*{Requirements }


\begin{DoxyItemize}
\item A C++ compiler which supports C++14.
\item An analytic signed distance function or a Watertight and orientable surface (only P\+LY files currently supported).
\end{DoxyItemize}

\subsection*{Basic usage }

\hyperlink{namespaceEBGeometry}{E\+B\+Geometry} is a header-\/only library in C++. To use it, simply make \hyperlink{EBGeometry_8hpp_source}{E\+B\+Geometry.\+hpp} visible to your code and include it.

To clone \hyperlink{namespaceEBGeometry}{E\+B\+Geometry}\+: \begin{DoxyVerb}git clone git@github.com:rmrsk/EBGeometry.git
\end{DoxyVerb}


Various examples are given in the Examples folder. To run one of the examples, navigate to the example and compile and run it. E.\+g., \begin{DoxyVerb}cd Examples/Basic
g++ -O3 -std=c++14 main.cpp
./a.out porsche.ply
\end{DoxyVerb}


The examples take the following steps that are specific to \hyperlink{namespaceEBGeometry}{E\+B\+Geometry}\+:


\begin{DoxyEnumerate}
\item Define an analytic signed distance function or parse a surface mesh into a D\+C\+EL mesh object.
\item Partition using B\+V\+Hs.
\item Compute the signed distance function.
\end{DoxyEnumerate}

More complex examples that use Chombo or A\+M\+ReX will also include application-\/specific code.

\subsection*{Advanced usage }

For more advanced usage, users can supply their own file parsers (only P\+LY files are currently supported), provide their own bounding volumes, or their own \hyperlink{namespaceBVH}{B\+VH} partitioners. \hyperlink{namespaceEBGeometry}{E\+B\+Geometry} is not too strict about these things, and uses rigorous templating for ensuring that the \hyperlink{namespaceEBGeometry}{E\+B\+Geometry} functionality can be extended.

\subsection*{License }

See L\+I\+C\+E\+N\+SE and Copyright.\+txt for redistribution rights. 